%-- place any standard commands/environments here to get included in
%-- documents.  When you include this file, you should do it before
%-- the \begin{document} tag.

%%%%%%%%%%%%%%%%%%%%%%%%%%%%%%%%%%%%%%%%%%%%%%%%%%%%%%%%%%%%%%%%%%%%%%
%-- CHANGES:
%-- 07/31/01 -jstrunk- Added command to set the paper margins.

%-- Provides fixed width font for commands and code snips.
\newcommand{\code}[1]{\texttt{\textbf{#1}}}

%-- Terms...  Use this to introduce a term in the paper.
\newcommand{\term}[1]{\emph{#1}}

%-- Provides stylization for e-mail addresses
% \newcommand{\email}[1]{\emph{(#1)}}

%-- Starts a minor section (puts the title inline w/ the text.
\newcommand{\minorsection}[1]{\textbf{#1}:}

%-- Jiri caption
\newcommand{\minicaption}[2]{\caption[#1]{\textbf{#1.} #2}}

%-- Units on numbers: 4KB -> \units{4}{KB}
\newcommand{\units}[2]{#1~#2}

%-- Commands...  i.e. WRITE commands.
\newcommand{\command}[1]{{\sc \MakeLowercase{#1}}}

%-- For notes about things that need to be fixed.
\newcommand{\fix}[1]{\marginpar{\LARGE\ensuremath{\bullet}}
    \MakeUppercase{\textbf{[#1]}}}
%-- For adding inline notes to a draft preceded by your initials
%-- E.g., \fixnote{JJW}{What the heck is a foobar?}
\newcommand{\fixnote}[2]{\marginpar{\LARGE\ensuremath{\bullet}}
    {\textbf{[#1:} \textit{#2\,}\textbf{]}}}

%-- Setting margins: \setmargins{left}{right}{top}{bottom}
\newcommand{\setmargins}[4]{
    % Calculations of top & bottom margins
    \setlength\topmargin{#3}
    \addtolength\topmargin{-.5in}  %-- seems like this should be 1, but .5
                                   %-- balances the text top to bottom
    \addtolength\topmargin{-\headheight}
    \addtolength\topmargin{-\headsep}
    \setlength\textheight{\paperheight}
    \addtolength\textheight{-#3}
    \addtolength\textheight{-#4}

    % Calculations of left & right margins
    \setlength\oddsidemargin{#1}
    \addtolength\oddsidemargin{-1in}
    \setlength\evensidemargin{\oddsidemargin}
    \setlength\textwidth{\paperwidth}
    \addtolength\textwidth{-#1}
    \addtolength\textwidth{-#2}
}

%-- For the tabularx environment... Using L, C, R as the column type
%-- will left, center, or right justify the text.
\newcolumntype{L}{X}
\newcolumntype{C}{>{\centering\arraybackslash}X}
\newcolumntype{R}{>{\raggedleft\arraybackslash}X}

%-- To comment out a swatch of text, use \omitit{blah blah blah}
\long\def\omitit#1{}

%-- For words tha are anonymized
\newcommand{\anonymized}[1]{\textit{**anonymized**}}

%-- For missing data that needs to be filled in
\newcounter{missingctr}
\newcommand{\missingval}[1]{\stepcounter{missingctr}\underline{\textbf{MISSING\_VAL~\arabic{missingctr}}}}

%-- For misisng data that need only be approximate
\newcounter{missingctrapprox}
\newcommand{\missingapprox}[1]{\stepcounter{missingctrapprox}\underline{\textbf{MISSING\_APPROX~\arabic{missingctrapprox}}}}

%-- Inline title; useful for sub-sub-sections in which you don't want a separate
%-- line for the title.
\newcommand{\inlinesection}[1]{\smallskip\noindent{\textbf{#1.}}}

\newenvironment{outlineenv}{\par\color{teal}}{\par}
\newenvironment{pagelenenv}{\par\color{red}}{\par}

\newcommand{\outline}[1]{\begin{outlineenv}#1\end{outlineenv}}
\newcommand{\pagelenblah}[1]{\begin{pagelenenv}#1\end{pagelenenv}}


\newcommand{\pagelen}[1]{
   \ifthenelse{\equal{\showcomments}{1}}{
     \pagelenblah{#1}}{}}

\newcommand{\outlinetext}[1]{
   \ifthenelse{\equal{\showcomments}{1}}{
     \outline{#1}}{}}

\newcounter{todoctr}
\newcounter{authoractr}
\newcounter{authorbctr}
\newcounter{authorcctr}
\newcounter{reviewerctr}


\newcommand{\todo}[1]{%
     \ifthenelse{\equal{\showcomments}{1}}{% true case
	\stepcounter{todoctr}%
	\textcolor{red}{\textbf{TODO~\arabic{todoctr}: #1}}
     }
     {} % false case
}
\newcommand{\mania}[1]{%
    \ifthenelse{\equal{\showcomments}{1}}{% true case
	\stepcounter{authoractr}%
	\textcolor{orange}{(Darby~\arabic{authoractr}: #1)}
    }
    {} % false case
}
\newcommand{\peter}[1]{%
     \ifthenelse{\equal{\showcomments}{1}}{% true case
	\stepcounter{authorbctr}%
	\textcolor{blue}{(Raja~\arabic{authorbctr}: #1)}%
     }
     {} % false case
}

%\newcommand{\raja}[1]{%
%     \ifthenelse{\equal{\showcomments}{1}}{% true case
%	\stepcounter{authorcctr}%
%	\textcolor{purple}{(Zhaoqi~\arabic{authorcctr}: #1)}%
%     }
%     {} % false case
%}

\newcommand{\reviewer}[2]{%
     \ifthenelse{\equal{\showcomments}{1}}{% true case
	\stepcounter{reviewerctr}%
	\textcolor{green}{reviewer #1: ~\arabic{reviewerctr}: #2}%
     }
     {} % false case
}


% -- Nice section names
\renewcommand{\sectionautorefname}{\S}

% Itmemize environment that is more space-efficient
\newenvironment{packeditemize}{
\begin{list}{$\bullet$}{
\setlength{\itemsep}{8pt}
\addtolength{\labelwidth}{4pt}
\setlength{\leftmargin}{0pt}%\labelwidth}
\addtolength{\leftmargin}{0pt}%\labelsep}
\setlength{\listparindent}{0pt}%\parindent}
\setlength{\parsep}{0pt}
\setlength{\topsep}{3pt}}}{\end{list}}

% an enumerate environment that is more space-efficient
\newcounter{packednmbr}
\newenvironment{packedenumerate}{
\begin{list}{\thepackednmbr.}{
\usecounter{packednmbr}
\setlength{\itemsep}{0pt}
\addtolength{\labelwidth}{0pt}%4pt}
\setlength{\leftmargin}{14pt}
\addtolength{\leftmargin}{0pt}%\labelsep}
\setlength{\listparindent}{0pt}%\parindent}
\setlength{\parsep}{0pt}
\setlength{\topsep}{3pt}}}{\end{list}}

% More space efficient items? Maybe needs to be used with packed itemize
\newenvironment{smallitem}{\setlength{\topsep}{0.0 truein}
   \begin{itemize}
   \setlength{\leftmargin}{.1 truein}
   \setlength{\parsep}{0.0 truein}
   \setlength{\parskip}{0.0 truein}
   \setlength{\itemsep}{0.0 truein}}{\end{itemize}}

% More spacefficient numbers?  Maybe needs to be used with packed enumerate
\newenvironment{smallenum}{\setlength{\topsep}{0.0 truein}
   \begin{enumerate}
   \setlength{\leftmargin}{.1 truein}
   \setlength{\parsep}{0.0 truein}
   \setlength{\parskip}{0.0 truein}
   \setlength{\itemsep}{0.0 truein}}{\end{enumerate}}

% Add requirements
\newcounter{RQnumber}
\newenvironment{RQenumerate}{
\begin{list}{\bf RQ\theRQnumber:}{
\usecounter{RQnumber}
\setlength{\itemsep}{1ex}
\addtolength{\labelwidth}{4pt}
\setlength{\leftmargin}{\labelwidth}
\addtolength{\leftmargin}{\parindent}
\setlength{\listparindent}{\parindent}
\setlength{\parsep}{0pt}
\setlength{\topsep}{3pt}}}{\end{list}}


% Nice color definitions for the below commands
\definecolor{Orange}{rgb}{1,0.5,0}
\definecolor{Red}{rgb}{1,0,0}
\definecolor{Purple}{rgb}{1,0,1}
\definecolor{Green}{rgb}{0,0.68,0}

\newcommand*\circled[1]{\tikz[baseline=(char.base)]{
            \node[shape=circle,draw,inner sep=1pt] (char) {#1};}}

% Skip a small amount of text
\newcommand{\Hair}{\ifmmode\mskip1mu\else\kern0.08em\fi}

% EN Dash
\newcommand{\mdash}{\unskip\Hair---\Hair\ignorespaces}

% Add period after an argument
\newcommand{\periodafter}[1]{#1.}

% Small paragraphs
\newcommand{\fakepara}[1]{\vspace{2mm}\noindent\textbf{#1}\hspace{2mm}}

% Colorred number for inline items
\newcommand{\itm}[1]{\item[(\textcolor{purple}{\textbf{#1}})]}

% Colored numer for inline items, emhasized
\newcommand{\iitm}[1]{\textit{(#1)}}

% Opportunity
\newcommand{\opp}[1]{(\textcolor{green!80!black!90}{\textbf{#1}})}
\newcommand{\op}[2]{\fakepara{\opp{#1} \textbf{#2}}}

% ????
\newcommand{\pr}[2]{\fakepara{(\textcolor{purple}{\textbf{#1}}) \textbf{#2}}}

% Challenge
\newcommand{\chall}[1]{(\textcolor{purple}{\textbf{Challenge #1}})}

% Insight
\newcommand{\insight}[1]{(\textcolor{green!80!black!90}{\textbf{Insight #1}})}

% Work Package
\newcommand{\wpkg}[1]{(\textcolor{blue!50}{\textbf{WP#1}})}

% Ways to add common things without Latex throwinga fit
\newcommand{\eg}{\textit{e.g.},\xspace}
\newcommand{\ie}{\textit{i.e.},\xspace}
\newcommand{\cf}{cf.\xspace}

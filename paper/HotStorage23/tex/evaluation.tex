\section{Evaluation}
\label{sec:evaluation}

Our prototype of ObjectFS is a FUSE filesystem implemented in roughly 2000
lines of C++. It performs garbage collection and checkpointing, but does not
yet implement snapshopts, cloning, overlays, or partial checkpoints.

In order to evaluate performance improvements in our filesystem, we evaluate
ObjectFS on a set of micro-I/O benchmarks workloads with Filebench
\citep{filebench} and compare its performance with S3FS and JuiceFS.

\subsection{Experimental Setup}

All three filesystems were mounted on a sigle machine within a cluster with a
Ceph RGW instance as the backend. The backend store is emptied of all objects
and the filesystems re-created between each run.

... something about the machine specs and backend setup

\subsection{Filebench results}

We evaluate all three filesystems on four filebench workloads: fileserver,
webserver, varmail, and random-write.

\begin{table}
	\caption{Total op/s comparison for each workload}
	\label{dataset}
	\centering
	\begin{tabular}{lrrrr}
		\toprule
		Workload     & ObFS  & S3FS & JuiceFS \\
		\midrule
		Fileserver   & 1745  & 670  & ---     \\
		Webserver    & 2385  & 1547 & ---     \\
		Varmail      & 2634  & 1150 & ---     \\
		Random write & 11001 & 463  & ---     \\
		\bottomrule
	\end{tabular}
\end{table}

\begin{table}
	\caption{Cluster Ceph RGWs}
	\label{rgws}
	\centering
	\begin{tabular}{lrrrr}
		\toprule
		Workload     & ObFS  & S3FS & JuiceFS \\
		\midrule
		Fileserver   & 1745  & 670  & ---     \\
		Webserver    & 2385  & 1547 & ---     \\
		Varmail      & 2634  & 1150 & ---     \\
		Random write & 11001 & 463  & ---     \\
		\bottomrule
	\end{tabular}
\end{table}







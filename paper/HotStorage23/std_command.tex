%%%%%%%%%%%%%%%%%%%%%%%%%%%%%%%%%%%%%%%%%%%%%%%%%%%%%%%%%%%%%%%%%%%%%%
%%%%%%%%%%% packages %%%%%%%%%%5
%%%%%%%%%%%%%%%%%%%%%%%%%%%%%%%%%%%%%%%%%%%%%%%%%%%%%%%%%%%%%%%%%%%%%%

\usepackage{tikz}
\usepackage{amsmath, amsfonts}
\usepackage{filecontents}
\usepackage[utf8]{inputenc}
\usepackage{graphicx}
\usepackage{booktabs}
\usepackage{url}
\usepackage{xspace}
\usepackage{microtype}
\usepackage{balance}
% \usepackage{cite}
\usepackage{cleveref}
\usepackage{caption}
\usepackage{subcaption}
\usepackage{flushend}
\usepackage{hyperref}
\usepackage[ruled,vlined]{algorithm2e}
\usepackage[normalem]{ulem}
\usepackage{booktabs}
\usepackage{wrapfig}
\usepackage{flushend}
\usepackage{listings}
\usepackage{float}
\usepackage{enumitem}
\usepackage{comment}
\usepackage{ulem}

\usepackage[T1]{fontenc}
\usepackage[italian]{babel}

%\usepackage{outlines}

\usepackage{blindtext}
\usepackage{etoolbox}


%%%%%%%%%%%%%%%%%%%%%%%%%%%%%%%%%%%%%%%%%%%%%%%%%%%%%%%%%%%%%%%%%%%%%%
%%%%%%%%%%% auxiliarty commands  %%%%%%%%%%5
%%%%%%%%%%%%%%%%%%%%%%%%%%%%%%%%%%%%%%%%%%%%%%%%%%%%%%%%%%%%%%%%%%%%%%


\newcommand{\showcomments}{0}

%-- Provides fixed width font for commands and code snips.
\newcommand{\code}[1]{\texttt{\textbf{#1}}}

%-- Terms...  Use this to introduce a term in the paper.
\newcommand{\term}[1]{\emph{#1}}

%-- Provides stylization for e-mail addresses
%\newcommand{\email}[1]{\emph{(#1)}}

%-- Starts a minor section (puts the title inline w/ the text.
%\newcommand{\minorsection}[1]{\noindent\textbf{#1}:}
\newcommand{\minorsection}[1]{\vspace{\smallskipamount}\noindent\textbf{#1. }}

%-- Jiri caption
\newcommand{\minicaption}[2]{\caption[#1]{\textbf{#1.} #2}}

%-- Units on numbers: 4KB -> \units{4}{KB}
\newcommand{\units}[2]{#1~#2}

%-- Commands...  i.e. WRITE commands.
\newcommand{\command}[1]{{\sc \MakeLowercase{#1}}}

%-- For notes about things that need to be fixed.
\newcommand{\fix}[1]{\marginpar{\LARGE\ensuremath{\bullet}}
    \MakeUppercase{\textbf{[#1]}}}
%-- For adding inline notes to a draft preceded by your initials
%-- E.g., \fixnote{JJW}{What the heck is a foobar?}
\newcommand{\fixnote}[2]{\marginpar{\LARGE\ensuremath{\bullet}}
    {\textbf{[#1:} \textit{#2\,}\textbf{]}}}

%-- Setting margins: \setmargins{left}{right}{top}{bottom}
\newcommand{\setmargins}[4]{
    % Calculations of top & bottom margins
    \setlength\topmargin{#3}
    \addtolength\topmargin{-.5in}  %-- seems like this should be 1, but .5
                                   %-- balances the text top to bottom
    \addtolength\topmargin{-\headheight}
    \addtolength\topmargin{-\headsep}
    \setlength\textheight{\paperheight}
    \addtolength\textheight{-#3}
    \addtolength\textheight{-#4}

    % Calculations of left & right margins
    \setlength\oddsidemargin{#1}
    \addtolength\oddsidemargin{-1in}
    \setlength\evensidemargin{\oddsidemargin}
    \setlength\textwidth{\paperwidth}
    \addtolength\textwidth{-#1}
    \addtolength\textwidth{-#2}
}

%-- For the tabularx environment... Using L, C, R as the column type
%-- will left, center, or right justify the text.

%-- To comment out a swatch of text, use \omitit{blah blah blah}
\long\def\omitit#1{}

%-- Inline title; useful for sub-sub-sections in which you don't want a separate
%-- line for the title.
\newcommand{\inlinesection}[1]{\smallskip\noindent{\textbf{#1.}}}

%-- todo notes

\newenvironment{outlineenv}{\par\color{red}}{\par}
\newcommand{\outline}[1]{\begin{outlineenv}#1\end{outlineenv}}

\newcommand{\todoO}[1]{\textsf{\textbf{\textcolor{Orange}{[[#1]]}}}}
\newcommand{\todoR}[1]{\textsf{\textbf{\textcolor{Red}{[#1]}}}}
\newcommand{\todoU}[1]{\textsf{\textbf{\textcolor{purple}{[[#1]]}}}}
\newcommand{\todoA}[1]{\textsf{\textbf{\textcolor{Yellow}{[[#1]]}}}}
\newcommand{\todoP}[1]{\textsf{\textbf{\textcolor{violet}{[[#1]]}}}}
\newcommand{\todoL}[1]{\textsf{\textbf{\textcolor{Brown}{[[#1]]}}}}
\newcommand{\todoT}[1]{\textsf{\textbf{\textcolor{Green}{[[#1]]}}}}
\newcommand{\outlinetext}[1]{
   \ifthenelse{\equal{\showcomments}{1}}{
      \outline{#1}}{}}
\newcommand{\oyk}[1]{
   \ifthenelse{\equal{\showcomments}{1}}{
      \todoR{oyk: #1}}{}}


\newcommand{\naive}{}% To make sure that \naive isn't already defined    
\def\naive/{na\"{\i}ve}

\newcommand{\para}[1]{\smallskip\noindent{\textbf{#1}}}



% for shepherding process
%\newcommand{\cradded}[1]{\textcolor{blue}{#1}}
% Cross out text
%\newcommand{\crdeleted}[1]{\textcolor{blue}{\sout{#1}}}
% Replace the second argument by the first argument
%\newcommand{\crreplaced}[2]{\textcolor{blue}{\sout{#2}}\textcolor{blue}{#1}}

% for camera ready process
\newcommand{\cradded}[1]{#1}
\newcommand{\crdeleted}[1]{}
\newcommand{\crreplaced}[2]{#1}

% Commands
\newcommand{\myparagraph}[1]{\vspace{\smallskipamount}\noindent\textbf{#1. }}
\newcommand{\eg}{\emph{e.g.}\xspace}
\newcommand{\etc}{etc.\@\xspace}
\newcommand{\cf}{{cf.}\xspace}
\newcommand{\ie}{\emph{i.e.}\xspace}
\newcommand{\etal}{\emph{et al.}\xspace}

\newcommand{\anonAzure}{a large cloud provider\xspace}

\newcommand{\anonFunctions}{a large serverless provider\xspace}

\newcommand{\sys}{Palette\xspace}

% Comment out toggletrue line to generate paper with no comments
\newtoggle{showmarks}
\toggletrue{showmarks}  %%%    <--- comment out this line :)


% side comment
\def\hn{\sffamily\selectfont}
\newcommand{\mpfont}{\hn\scriptsize}
\newcommand{\MPworker}[2]{\unskip{\color{#1}\vrule\vrule}{\marginpar{\raggedright\color{#1}\mpfont #2}}}
\iftoggle{showmarks}{
  \newcommand\mania[1]{\textcolor{magenta}{mania: #1}}
  \newcommand\dsb[1]{\textcolor{green}{DB: #1}}
  \newcommand\rf[1]{\textcolor{cyan}{RF: #1}}
  \newcommand\inigo[1]{\textcolor{red}{IG: #1}}
  \newcommand\gohar[1]{\textcolor{orange}{GOHAR: #1}}
  \newcommand\cheng[1]{\textcolor{blue}{cheng: #1}}
  \newcommand\jiyong[1]{\textcolor{orange}{jy: #1}}
  \newcommand\TODO[1]{\textcolor{blue}{TODO: #1}}

  \newcommand{\CP}[1]{\MPworker{blue}{CT: #1}}
}{
  \newcommand\mania[1]{\unskip}
  \newcommand\daniel[1]{\unskip}
  \newcommand\rf[1]{\unskip}
  \newcommand\inigo[1]{\unskip}
  \newcommand\gohar[1]{\unskip}
  \newcommand\cheng[1]{\unskip}
  \newcommand\jiyong[1]{\unskip}
  \newcommand\TODO[1]{\unskip}

  \newcommand{\CP}[1]{\unskip}
}

\newcommand{\heading}[1]{
  \vspace{1ex}
  \noindent
  \textbf{#1}}

\newcommand{\todo}[1]{{\color{red}#1}}

% names
\newcommand{\sqlite}{SQLite\xspace}
\newcommand{\pmem}{PMem\xspace}
\newcommand{\tuner}{configure tuner\xspace}


